% !TeX root = these.tex


\chapter*{Introduction}
%étapes traditionnellement dans l’introduction :
%• l’accroche, 1) capter l’attention du lecteur
%• la problématique 2) délimiter le sujet
%• l’annonce du plan (très claire) 3) annoncer le plan

Dans le domaine de l'enseignement, la création d'horaires de cours est une tâche ardue pour les secrétariats. En effet, il s'agit de prendre simultanément en compte de très nombreuses contraintes ; les locaux ne sont pas toujours libres, les professeurs ont des \textit{desiderata}, les élèves ont des parcours différents. Cette somme de contraintes font de la mise en place d'horaires un travail long et fastidieux qu'il s'agit de réitérer à chaque rentrée scolaire. 
\newline
\indent
Ce type de problème est un problème récurrent dans le domaine informatique.

Face à ce constat, de nombreux travaux ont eu pour sujet la \enquote{constraint programming}. 
\newline
\indent
L'objectif de notre mémoire est d'aider à la création des horaires de l'EPHEC.
Dans cet objectif, nous avons décidé de créer une application internet permettant d'accéder
aux données présentes sur les serveurs de l'école.\\
\newline
\indent
Cette application est créée pour être utilisée au sein de l'école, c'est pourquoi il faut un 
outil adapté à l'EPHEC simple d'utilisation et interactif. \\
\newline
\indent
Ce rapport décrira la méthodologie utilisée et les choix de conception faits.
Par la suite, nous décrirons plus profondément le programme et sa structure avant de clore sur une discussion sur les possibilités du programme. 



