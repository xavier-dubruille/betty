\section{Discussions}
\subsection{Choix concernant la performance}
\subsection{Choix concernant la sécurité}

Du fait de nos choix de conception, tel que gwt pour générer le java-script, ou
hibernate pour abstraire la base de donnée, ou le choix des rpc pour la
commnuniquation client-serveur,..  Nous pensons avoir créé une application ``de
base'', très sécure, et qu'il serait tout à fait envisageable de faire tourner notre
application sur un serveur externe.
(xxx) xss, csrf, sha256, check (basique) de la complexité du mot de passe, ainsi
qu'un catcha très basique aussi pour eviter des bots.

La secu du coté serveur à également été pris en compte, mais, logiquement, en
moins poussé (quoi que).  A savoir: l'application tourne sur une debian *stable*
(à savoir Squeeze), n'ayant que peu de services installé.
Tomcat, notre conteneur d'applet, est lui aussi à jours, et en version stable,
il ne bénificie d'aucun droit root, ainsi donc, (contrairement à une autre
application tournant sur Windows par exemple), il n'a pas le droit d'emmetre sur
le port 80. Pour garder l'application safe et peformante, une redirection de
son port d'origine est effectué par netfiller sur le port 80.

Nous avons pas poussé plus loins la sécu du coté serveur, étant donnée qu'elle
est destiné à fonctionner sur d'autre machine, mais en situation réel, il serait
jusdicieu de mettre en place un liens https.  En situation réel, nous
préconisons xxx (parfeu, strapping, vm,..)

N'étant pas infaillble, la mise à jour (très aisée pour notre type
d'application), est nous semple -t-il, une part importante dans la sécurité du
système. (cela probablement accentué par la mise de notre code source sous une
license ouverte (gpl3) )
\subsection{Possibilités d'amélioration du programme}

\subsubsection{mode comparaison}
C'était un de nos objectifs principaux; il était question de pouvoir créer à la
voilée les collonnes représentant plusieur professeurs, loacux ou classe, tout
en gardant, pour les lignes, les périodes. Malheureusement le manque de temps à
eu raison de nous, ains qu'un problème fondamentale, à savoire qu'il n'es pas
possible, de rendre une région ne fesant pas partie du DOM (typiquement rajouté
en js, donc d'une manière non-statique) comme étant une région ``dropable''.
Cela n'étant certainement pas impossible, la solution idéal nous est pas encore
connue; probablement qui faudrait rendre l'entièreté de la région contenant le
tableau comme ``dropable'' afin de subdifiser celle ci.. car actuelement ce soit
les ``cases'' qui sont dropable..
\subsubsection{solveur}
Ce fu également un point assez frustrant, étant donné nos recheches abondante
sur le sujet, et surtout la trouvaille de cette librairire ``miraculeuse'' écrit
par Muller, et qui, en plus d'être en gpl et fortement maintenue (dernière
version date d'il y a 2 semaines), elle nous parrait très performante et d'une
rare adéquoicité.  C'est donc avec bcp de tristesse dans l'ame (:p), qu'on n'en
utilise qu'une fractions de ses possibilité.  C'était tout l'avantage de notre
application (et de sont architectur client-server) et ce n'est pas utilisé a son
full potentiel.  C'est donc la première amérioration que devra subir notre
application, car tout est en place pour pouvoir en tirer un avantage certain.
\subsubsection{droits}
Un des but de cette architecture était de pouvoir avoir plusieurs compte lié à
un projet, avec des droits différents, imaginons des professeurs qui pouraient
éditer leur désidérata, ou certaine parsonne qui pourait suivre l'évoltion de la
construction de l'horaire sans pour autant avoir les droits de modification.
\subsubsection{mode hors ligne}
On se base entièrement sur le local storage, les requètes vers le serveur ne
servent qu'a se connecter, charger le projet et enregistrer des modifications.
Actuelement, il est possible de continuer la création d'un horraire chargé en
mémoire sans la necessité d'une connections, cependant, pour pouvoir pleinement
utiliser cette fonctionnalité, il est necessaire de rajouter certaine xxx, tel
qu'une resyncronisation des modifications faite en local avec la bdd lors que la
connection est retrouvée ainsi que la possibilité de reprendre un projet en
cache.
\subsubsection{upload à partir d'une bdd}
\subsubsection{modifications de données}
actuelement il n'est pas possible de modifier toutes les données intervenant
dans l'application (notament la création d'un prof, local, etc), 
