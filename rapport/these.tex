\documentclass{these}
\usepackage[french]{babel}
\usepackage[utf8]{inputenc} 
\usepackage[T1]{fontenc}
\usepackage{lmodern}
\usepackage[babel=true]{csquotes}
\usepackage{lscape}
\usepackage{amssymb}
\usepackage{chngpage}
\usepackage{chngcntr}

% Informations factuelles
\title{\huge Analyse et intégration de techniques \enquote{constraint programming} et \enquote{operations research} pour la réalisation d'un outil d'aide à la création d'horaires}
\author{Kevin \textsc{Jacoby} et Xavier \textsc{Dubruille}}
\date{2012}
\labo{CENTAL}
\school{École Doctorale}
\speciality{Traitement Automatique du Langage}
\university{Linguistique à Finalité TAL}{Ecole Pratique des Hautes Etudes Commerciales}
%\ISBN{} %à remplir et décommenter lorsque ce numéro est connu.
\advisor[M]{C. \bsc{Lambeau}} %Directeur de thèse


% Compteur de figure
\counterwithout{figure}{chapter}
\counterwithout{table}{chapter}


% Profondeur TOC
\setcounter{tocdepth}{1}
\stepcounter{secnumdepth}
\stepcounter{tocdepth}


% Références
\usepackage[colorlinks=false]{hyperref}
\hypersetup{urlcolor=black,linkcolor=black,citecolor=black,colorlinks=true}
%\usepackage[pdftex, pdfborder={0 0 0},hypertex]{hyperref}

% Math
\everymath{\displaystyle}

% Code
\usepackage[dvipsnames]{xcolor}
\usepackage{listings}
\lstset{% general command to set parameter(s)
language=SQL,
basicstyle=\small,
% print whole listing small
keywordstyle=\color{violet}\bfseries\bf,
% underlined bold black keywords
identifierstyle=\color{blue},
% nothing happens
commentstyle=\color{red}\bf, % white comments
stringstyle=\ttfamily,
% typewriter type for strings
showstringspaces=false}

% Pstree
%\usepackage{pstricks}
%\usepackage{pst-plot,pst-text,pst-tree,pst-eps,pst-fill,pst-node,pst-math}
%\usepackage[on]{auto-pst-pdf} 
%\def\TTR{\Tr{\Tcircle[fillstyle=solid,fillcolor=black!60]%
%  {}}}
  
% Paragraph
\makeatletter
\renewcommand{\paragraph}{\@startsection{paragraph}{4}{0ex}%
   {-3.25ex plus -1ex minus -0.2ex}%
   {1.5ex plus 0.2ex}%
   {\normalfont\normalsize\bfseries}}
\makeatother

% Page vierge après Part
\makeatletter
\renewcommand{\@endpart}{\vfil\newpage\if@tempswa\twocolumn\fi}
\makeatother

%Nospace
\makeatletter
\@ifpackageloaded{babel}%
        {\newcommand{\nospace}[1]{{\NoAutoSpaceBeforeFDP{}#1}}}%  % !! double {{}} pour cantonner l'effet à l'argument #1 !!
        {\newcommand{\nospace}[1]{#1}}
\makeatother

% Citation - Bibliographie
\usepackage{natbib}
\bibpunct{(}{)}{;}{a}{}{;} 

% Intégration d'images
\usepackage{graphicx}
%\graphicspath{{./pics/}} 

% Siècle
\def\siecle#1{\textsc{\romannumeral #1}\textsuperscript{e}~siècle}

%%%%%%%%%%%%%%%%%%%%%%%%%%%%%%
% Document
%%%%%%%%%%%%%%%%%%%%%%%%%%%%%%
\begin{document}

% Titre
\frontmatter
\maketitle


%\maketitle
%\setcounter{page}{6} 

%%%%%%%%%%%%%%%%%%%%%%%%%%%%%%
% Resume
%%%%%%%%%%%%%%%%%%%%%%%%%%%%%%
\input{resume.tex}
%%%%%%%%%%%%%%%%%%%%%%%%%%%%%%
% Remerciements
%%%%%%%%%%%%%%%%%%%%%%%%%%%%%%
% !TeX root = these.tex

\newpage
\chapter*{Avant-propos}
En préambule à ce travail de fin d'études, nous souhaitons adresser nos remerciements les plus sincères aux personnes qui nous ont apporté
leur aide et qui ont contribué à l'élaboration de ce mémoire.\\
\newline
\indent
Nous tenons à remercier chaleureusement:\\
\begin{itemize}
\item Monsieur Lambeau, notre promoteur, pour le temps qu'il a bien voulu nous consacrer et sans qui ce mémoire n'aurait jamais vu le jour\\
\item Monsieur Pierre \textsc{Schaus}, pour ses explications sur les problèmes de satisfaction de contraintes\\
\item Monsieur Fauconnier, pour la grande patience dont il a su faire preuve et le temps précieux qu'il nous a accordé\\
%      \item Monsieur François Henry, notre assistant en thermodynamique, qui a accepté de répondre à nos questions avec gentillesse,
\item Madame Bonnave qui a eu la gentillesse de relire et corriger ce travail\\

\item Monsieur Tonon, pour la correction de ce travail.
\end{itemize}

%%%%%%%%%%%%%%%%%%%%%%%%%%%%%%
% TOC
%%%%%%%%%%%%%%%%%%%%%%%%%%%%%%
\newpage
\tableofcontents

%%%%%%%%%%%%%%%%%%%%%%%%%%%%%%
% Abréviation
%%%%%%%%%%%%%%%%%%%%%%%%%%%%%%
\newpage
%% !TeX root = these.tex

\chapter*{Liste des abréviations}

\begin{tabular}{l l}
\textbf{ASCII} & American Standard Code for Information Interchange\\
\textbf{CERN} & European Organization for Nuclear Research\\
\textbf{DEC} & Digital Equipment Corporation\\
\textbf{DNS} & Domain Name System\\
\textbf{DTD} & Document Type Definition\\
\textbf{FAI} & Fournisseur d'Accès à Internet\\
\textbf{FTP} & File Transfer Protocol\\
\textbf{HTTP} & Hypertext Transfer Protocol\\
\textbf{IA} & Intelligence Artificielle\\
\textbf{IBM} & International Business Machines\\
\textbf{ICANN} & Internet Corporation for Assigned Names and Numbers\\
\textbf{IC} & Ingénierie des Connaissances\\
\textbf{IETF} & Internet Engineering Task Force\\
\textbf{IRI} & Internationalized Resource Identifier\\
\textbf{ISP} & Internet Service Provider\\
\textbf{OWL} & Web Ontology Language\\
\textbf{RDF} & Resource Description Framework\\
\textbf{RDFS} & RDF Schema\\
\textbf{RFC} & Requests for Comments\\
\textbf{RI} & Recherche d'Information\\
\textbf{SBC} & Système à Base de Connaissances\\
\textbf{SGBD} & Système de Gestion de Base de Données\\
\textbf{SGML} & Standard Generalized Markup Language\\
\textbf{SPARQL} & SPARQL Protocol and RDF Query Language\\
\textbf{SQL} & Structured Query Language\\
\textbf{SRI} & Système de Recherche d'Information\\
\textbf{TREC} & Text REtrieval Conference\\
\textbf{UML} & Unified Modeling Language\\
\textbf{URI} & Uniform Resource Identifier\\
\textbf{URL} & Uniform Resource Locator\\
\textbf{URN} & Uniform Resource Name\\
\textbf{W3C} & World Wide Web Consortium\\
\textbf{XML} & Extensible Markup Language\\
\end{tabular}








%%%%%%%%%%%%%%%%%%%%%%%%%%%%%%
% Debut du document
%%%%%%%%%%%%%%%%%%%%%%%%%%%%%%
\newpage
\mainmatter
% !TeX root = these.tex


\chapter*{Introduction}
%étapes traditionnellement dans l’introduction :
%• l’accroche, 1) capter l’attention du lecteur
%• la problématique 2) délimiter le sujet
%• l’annonce du plan (très claire) 3) annoncer le plan

Dans le domaine de l'enseignement, la création des horaires de cours est une tâche ardue pour les secrétariats. En effet, il s'agit de prendre simultanément en compte de très nombreuses contraintes ; les locaux ne sont pas toujours libres, les professeurs ont des \textit{desiderata} particuliers, les élèves ont des parcours différents, etc. Cette somme de contraintes font de la mise en place d'emploi du temps un travail long et fastidieux qu'il s'agit de réitérer à chaque rentrée scolaire.\\
\newline
\indent
Ce type de problème touche au domaine plus large de la programmation par contraintes, dite \enquote{constraint programming} \citep{Muller_2005}. Ce paradigme de programmation s'attache à traiter les problèmes où une large combinatoire est nécessaire pour trouver un ensemble de solutions satisfaisant les différents acteurs impliqués. Par exemple, la programmation par contraintes est notamment utilisée pour traiter les problèmes relatifs aux circuits d'attente aérien au-dessus des aéroports. Il s'agit de prendre simultanément en compte la quantité de kérosène restante pour chacun des avions, les pistes occupées ou en passe de l'être, les prédictions météos, les \enquote{gates} libres, les ressources disponibles au sol, etc.
\newline
\indent


une approche par contraintes est utilisée dans les domaines des transports,  la gestion de main d’œuvre, la , la création d'horaires dans les tournois sportifs, etc.\\

  Par exemple, le problème du jeu \textit{sudokou} est un exemple où la programmation par contraintes pourrait intervenir pour trouver une solution. Cependant, dans le \textit{sudokou}, la contrainte est simple ; un chiffre doit être unique dans la ligne et la colonne qu'il occupe.
\newline
\indent
La programmation par contraintes par contraintes est utilisées dans de nombreux domaines tels que les transports, la gestion de main d’œuvre, la création d'horaires dans les tournois sportifs, etc.\\
\newline
\indent

Face à ce constat, de nombreux travaux ont eu pour sujet la \enquote{constraint programming}. 
\newline
\indent
L'objectif de notre mémoire est d'aider à la création des horaires de l'EPHEC.
Dans cet objectif, nous avons décidé de créer une application internet permettant d'accéder
aux données présentes sur les serveurs de l'école.\\
\newline
\indent
Cette application est créée pour être utilisée au sein de l'école, c'est pourquoi il faut un 
outil adapté à l'EPHEC simple d'utilisation et interactif. \\
\newline
\indent
Ce rapport décrira la méthodologie utilisée et les choix de conception faits.
Par la suite, nous décrirons plus profondément le programme et sa structure avant de clore sur une discussion sur les possibilités du programme. 


Avant tout, il est nécessaire de bien saisir la problématique et en quoi celle-ci peut être utile à notre travail. Nous avons du faire un analyse approfondie sur ce qu'était une contraintes ainsi que la programmation de celles-ci. Il a fallu ensuite analyser les différents outils nous permettant de "programmer par contraintes" et ensuite essayer de les implémenter dans un logiciel d'aide à la création d'horaire que nous avons du créer au préalable.\\
\newline
\indent
%D'abord, pourquoi on a fait ça et pourquoi ce TFE. En quoi c'est utilite pr notre travail
Madame Gillet, Directrice de l'établissement Ephec à Louvain-la-Neuve, établit l'horaire a chaque nouveau quadrimestre de l'année scolaire. l'établissement d'un horaire doit se faire sous certaines contraintes et en fonction de désidératas remis par les professeurs. Le nombre de locaux informatique est limité, certains professeurs sont dit "externe" à l'Ephec, ceux-ci doivent se voir attribué un horaire particulier, certain cours se donne dans des locaux externes à l'Ephec et ne sont donc pas disponible à chaque période de cours. Toute ces contraintes, si elle ne sont pas informatisée doivent prise en compte par la personne établissant l'horaire qui doit réaliser un "vrai casse tête chinois" afin d'avoir un horaire le plus adéquat possible.\\
\newline
\indent
La programmation par contrainte permet de résoudre de manière automatique cette problématique, facilitant ainsi la tâche qui incombe à la personne en charge de l'élaboration d'un horaire. Nous allons d'abord définir ce qu'est une contraintes de manière plus théorique pour bien saisir la problématique et la provenance de celle-ci. Nous discuterons ensuite autour des différentes solutions existantes nous permettant de réaliser cela.




% analyse de la problematique ainsi que les objectifs
%% !TeX root = these.tex

\chapter{Analyse de la problématique et outils existants}
\section{La programmation par contraintes}
Avant tout, il est nécessaire de bien saisir la problématique et en quoi celle-ci peut être utile à notre travail. Nous avons du faire un analyse approfondie sur ce qu'était une contraintes ainsi que la programmation de celles-ci. Il a fallu ensuite analyser les différents outils nous permettant de "programmer par contraintes" et ensuite essayer de les implémenter dans un logiciel d'aide à la création d'horaire que nous avons du créer au préalable.\\
\newline
\indent
%D'abord, pourquoi on a fait ça et pourquoi ce TFE. En quoi c'est utilite pr notre travail
Madame Gillet, Directrice de l'établissement Ephec à Louvain-la-Neuve, établit l'horaire a chaque nouveau quadrimestre de l'année scolaire. l'établissement d'un horaire doit se faire sous certaines contraintes et en fonction de désidératas remis par les professeurs. Le nombre de locaux informatique est limité, certains professeurs sont dit "externe" à l'Ephec, ceux-ci doivent se voir attribué un horaire particulier, certain cours se donne dans des locaux externes à l'Ephec et ne sont donc pas disponible à chaque période de cours. Toute ces contraintes, si elle ne sont pas informatisée doivent prise en compte par la personne établissant l'horaire qui doit réaliser un "vrai casse tête chinois" afin d'avoir un horaire le plus adéquat possible.\\
\newline
\indent
La programmation par contrainte permet de résoudre de manière automatique cette problématique, facilitant ainsi la tâche qui incombe à la personne en charge de l'élaboration d'un horaire. Nous allons d'abord définir ce qu'est une contraintes de manière plus théorique pour bien saisir la problématique et la provenance de celle-ci. Nous discuterons ensuite autour des différentes solutions existantes nous permettant de réaliser cela.

%Parler de notre rencontre avec Pierre bidule je sais plus quoi
%je sais plus ce qu'on avais dit d'autre

% presentation de l'application (option disponible, comment ça marche,...)

% !TeX root = these.tex

\chapter{Présentation de l'application}
Notre application, surnommée Betty\footnote{Brillant Ephec Time Tabling for You}, est un outil permettant d'élaborer un horaire de façon plus conviviale et plus rapide comparé à la façon actuelle de procéder à l'Ephec.\\
\newline
\indent
Cependant, avant tout développement ultérieur, il est nécessaire de présenter les différents objets composant notre solution. Nous avons choisi de travailler par \texttt{projet}. Chaque projet est caractérisé par un ensemble de cours étalés sur deux semestres. Betty permet de gérer plusieurs projets. Dans une optique de confidentialité, l'accès à chaque projet nécessite une \texttt{inscription} et une \texttt{connexion} de la part de l'utilisateur.
\newline
\indent
Au sein de chaque projet, chaque cours est représenté par ce que nous appelons un \texttt{carton}. Un carton est un cadre décrivant le nom du cours, son sigle et le professeur qui lui est rattaché. Betty propose par un système intuitif de \enquote{drag and drop} de positionner ces cartons au sein d'un \texttt{semainier} afin de construire un horaire.
\newline
\indent
Sur cette page présentant le semainier, différents outils sont utilisables afin d'améliorer la création de l'horaire. À chaque cours sont rattachés ses \texttt{attributions} spécifiques, c'est-à-dire les différentes classes d'élèves auxquelles se destine le cours. Un système de \texttt{notifications} a été mis en place afin d'offrir à l'utilisateur un retour sur les opérations qu'il effectue. Enfin, un système de \texttt{filtres} améliore la vision des données. 
\newline
\indent
La programmation par contraintes est utilisée pour définir de nouveaux horaires. Cette utilisation des possibilités s'effectue au travers d'un panneau d'\texttt{instances}. Une instance  est un état spécifique des cartons, c'est-à-dire un horaire potentiel. Lorsque l'utilisateur l'estime nécessaire, il est possible d'utiliser le  \texttt{solveur}. Le solveur est le système qui est en charge de calculer un nouvel horaire, c'est-à-dire une nouvelle instance, répondant à certaines contraintes.
\newline
\indent
Afin de diminuer les temps de chargement, un mécanisme permet d'exploiter la \texttt{mémoire cache} des navigateurs côté client. Ce procédé permet de minimiser le nombre de requêtes envoyées au serveur et améliore la rapidité de l'ensemble du système.\\
\newline
\indent
Cette présente section décrira ces différents aspects.


\section{Connexion et Inscription}

La première page de l'application propose à l'utilisateur d'entrer son nom d'utilisateur et son mot de passe. Le mot de passe est crypté en SHA-256\footnote{Secure Hash Algorithm 256 bits}, dans la base de données. La page propose également de vous inscrire via la flèche en haut à droite de la fenêtre.

\begin{figure}[!h]
	\begin{center}
	\includegraphics[width=8cm,height=4cm]{login.png}	
	\includegraphics[width=8cm,height=4cm]{subscribe.png}
	\caption{page de login et d'inscription}
\end{center}
\end{figure}

Lors de l'inscription, nous invitons l'utilisateur à entrer sont nom d'utilisateur, mot de passe et un email. Les informations entrées sont soumises à des vérifications d'usage, le maximum de ces vérifications étant effectué du côté client afin de réduire l'échange avec le serveur.
\newline
\indent
Pour cette partie, quelques améliorations peuvent être apportées tel que le changement de mot de passe ou la récupération de celui-ci par envoi de mail.

\section{Page des projets}

La page est ordonnée de façon à avoir toujours le dernier projet en date créé en haut de la liste. Un projet est présenté de la manière la plus concise possible afin de ne pas perdre l'utilisateur. Ce dernier à la possibilité de créer un nouveau projet\footnote{Option également disponible via le menu}, choisir le semestre à élaborer et de supprimer un projet. L'application devrait dans une prochaine mise à jour proposer des options sur celui-ci tels que le partage de projets entre plusieurs utilisateurs.\\
\begin{figure}[!h]
	\begin{center}
	\includegraphics[width=12cm,height=4cm]{project.png}	
	\caption{représentation d'un projet}
\end{center}
\end{figure}
\\Lors de la création d'un nouveau projet, celui-ci doit être nommé et contenir les fichiers nécessaires à l'élaboration de l'horaire. Ces fichiers se présentent sous la forme d'un .xls contenant pour le premier, la liste des attributions de chaque cours, ce fichier est le résultat d'une requête SQL et nous a été fourni par l'Ephec. Le deuxième représente la liste des locaux disponibles, ce fichier n'existant pas en tant que tel, à initialement été créé par Madame Vroman, Professeur à l'Ephec, dans le cadre du "projet horaire" du cours de langage avancé de programmation de deuxième année. Nous y avons ajouté des informations pouvant être prises en compte par le solveur, et permettant de facilité l'établissement manuel d'un horaire.\\
\\
Une fois le projet créé et le quadrimestre sélectionné, l'utilisateur est redirigé vers la page principale de l'application dans laquelle l'horaire pourra être créé.

\section{Page principale}
La page principale se présente comme suit:\\

\begin{itemize}	
	
	\item Au nord de la page, nous trouvons les différents filtres et options disponibles tels que :
	\begin{enumerate}
		\item Card filter
		\item Sélection de la grille à afficher
		\item Un panneau regroupant les différentes instances du projet\\
	\end{enumerate}
	\item À l'ouest, les cartons créés sur base du fichier des attributions\\
	\item Au centre, le semainier où les cartons pourrons venir se glisser\\
	\item À l'est, un panneau de notifications permettant d'avoir un suivi des différentes actions faite par l'utilisateur.
\end{itemize}

\begin{figure}[!h]
	\begin{center}
	\includegraphics[width=16cm,height=6cm]{littlemain.png}	
	\caption{vue globale de la page principale}
\end{center}
\end{figure}

\subsection{Les attributions}
La mise en forme des cartons se base sur le design de ceux actuellement employés à l'Ephec lors de l'établissement manuel de l'horaire. Ceux-ci comportent le/les classe(s) assignée(s) à chaque cours donné par un professeur.\\
\newline
\indent
À chaque carton est assigné un ensemble de locaux où pourront se donner les cours. Par exemple, un carton de type informatique sera assigné uniquement aux locaux de type informatique. Lorsque le carton est déposé, la solution lui assigne un local disponible parmi cette liste, et nous pouvons voir apparaître le nom du local en bas à droite du carton.

\subsection{Semainier}
Nous affichons dans le semainier, les informations relatives à la personne, classe ou au local choisis via le filtre prévu à cet effet. La grille se colorie en fonction du carton qui est sélectionné. Un code de couleurs a été mis en place permettant de distinguer si un carton peut être placé ou pas. Par exemple, si l'on se trouve dans la vue d'une classe et que l'on prend un carton d'une autre classe, toute la grille se coloriera en rouge, montrant à l'utilisateur que celui-ci ne peut pas être placé.\\
\newline
\indent
Nous distinguons trois groupes de couleurs ; vert, orange et rouge. Ces groupes sont eux-mêmes subdivisés en trois catégories: couleur claire, normale et foncée. Lorsqu'on colorie la grille horaire, il arrive parfois qu'un carton puisse être placé à plusieurs endroits, les uns plus avantageux que d'autres pour la suite de l'établissement de l'horaire, il est donc nécessaire de pouvoir dire à l'utilisateur que le carton peut être placé, mais l'orienter sur un choix plus adéquat. Ce code de couleurs est utilisé dans une version limitée pour le moment.


\subsection{Notifications}
Les notifications sont un support pour l'utilisateur. Lorsque celui-ci effectue une action comme supprimer/ajouter un carton il est nécessaire de savoir si l'action c'est effectuée correctement. À la place d'un popup intrusif, nous avons opté pour ce système, signalant à l'utilisateur de manière plus douce l'état d'actions qui ne peuvent être vues via une interface graphique.\\
\newline
\indent
Nous distinguons ici deux couleurs différentes, une couleur se fondant au thème général de l'application lorsque tout s'est effectué correctement et une couleur rouge pour signaler un problème. Ce système est limité et pourra être amélioré dans le futur.


\subsection{Les filtres}
Les différents filtres permettent de faciliter la création de l'horaire de façon manuelle. Si nous voulons créer l'horaire d'un professeur en particulier, avoir les cartons de tous les professeurs rendrait la tâche plus lourde à l'utilisateur. L'utilisation de ce filtre permet d'avoir une meilleure vue sur ce qui doit être placé. Nous parlerons de vue \enquote{vue}.
\newline
\indent
De même, il est possible d'afficher/masquer les cartons déjà placés. Lorsqu'on filtre les cartons d'une classe et que tous ces cartons sont placés, ils ne font théoriquement plus partie de la liste des cartons et donc l'utilisateur n'a pas la possibilité de savoir si ladite classe possède des cartons. Ce système favorise aussi la vue d'ensemble sur ce qui a déjà ou non été placé.\\
\newline
\indent
Une dernière option est de pouvoir automatiquement \enquote{switcher} sur la grille horaire correspondant au filtre mis sur les cartons. Si cette option est sélectionnée et que nous filtrons les cartons de la classe \textit{3TL2}, nous supposons ici que l'objectif est de placer les attributions de cette classe. Par conséquent, seule la grille horaire des \textit{3TL2} est affichée.\\
\newline
\indent
L'application a donc été pensée afin de minimiser l'effort cognitif de l'utilisateur et de lui offrir les outils adéquats pour se concentrer sur son seul objectif ; la création de l'horaire.


\subsection{Les instances}

Le panneau d'instance permet, au sein d'un même projet, de créer un ensemble de 
sous-projets. L'objectif principal de ce panneau réside dans l'utilisation du solveur. Celui-ci sera lancé dans une nouvelle instance, permettant à l'utilisateur de continuer son horaire manuellement dans une autre sans être bloqué. Une fois la résolution finie, l'utilisateur peut naviguer entre les différentes instances afin de voir les différents résultats obtenus. \\
\newline
\indent
C'est ici que l'implémentation des instances y trouve sa principale utilité. Toutefois, il serait envisageable, comme perspective d'extension à ce travail, de pouvoir comparer deux instances entre elles.


%Cette partie est à améliorer et à modifier!
\subsection{Le solveur}
Pour lancer le solveur, nous devons aller dans le menu \texttt{project > solveur > solve}. Une fenêtre s'ouvre permettant à l'utilisateur de sélectionner l'instance dans laquelle doit s'effectuer la résolution. Une fois le solveur lancé, une notification apparaît à l'utilisateur lui notifiant que celui-ci est exécuté. À la fin de tentative de résolution, l'utilisateur reçoit une notification lui spécifiant la fin du travail et la bonne ou mauvaise application de ce dernier. Le solveur, dans sa version actuelle, fonctionne correctement, mais ne propose, à l'heure actuelle, aucune option de configuration et fonctionne sur des petits projets. 


\subsection{Mémoire cache}
L'application utilise la mémoire cache du navigateur pour stocker les données, ainsi nous minimisons les requêtes vers le serveur. Ceci pourrait être également exploité pour pouvoir travailler sur l'application sans connexion internet. Les données nécessaires à l'établissement de l'horaire étant stockées dans la partie cliente.





















% nos choix qd aux techno, gwt et autre
% !TeX root = these.tex

\chapter{Cadre technologique}

%************************ CLIENT SERVEUR *********************************
\section{Client/Serveur}
Notre application se base sur une architecture client serveur pour les multiples
avantages que celle-ci apporte.\\
Premièrement l'Ephec étant un établissement possédant plusieurs sites,
 une application client serveur peut permettre d'avoir un centre de données
 commun.
 De cette manière, un horaire établi à Louvain-la-Neuve pourra être pris en
 compte lors de l'établissement d'un horaire à Bruxelles.\\
L'avantage de partager la charge de travail, la grosse partie (solveur,…) étant
effectuée sur le serveur, permet que l'élaboration de l'horaire puisse se
faire sur des machines possédant peu de ressources. Par exemple, il sera
possible d'établir l'horaire sur une tablette ou encore sur un Smartphone.\\
\\
Grâce à ce type d'architecture, l'application n'est pas dépendante du système
d'exploitation mis en place ni de l'ordinateur. L'horaire peut être débuté sur
une machine Windows, pour ensuite être continuer sur une tablette.
Les mises à jour de l'application sont aussi complètement transparentes pour
l'utilisateur final. Pas besoin de télécharger et d'installer les mises à jour
comme sur un logiciel orienté desktop.\\
De même, en comparaison toujours avec une application desktop, si il survient un
problème avec la station de travail, il est garanti de pouvoir retrouver ces
données dans leurs entières intégralités, et l'horaire peut continuer à être établi
sur une autre station. Les serveurs offrent de nombreux avantages (duplication des
données, séparations sur plusieurs sites,…) qu'un ordinateur en panne ne peut offrir.\\
\\
Outre cet aspect de facilité pour la partie cliente, il est aussi très facile
d'administrer l'application. Celle-ci étant portable et facile d'installation.
Pas besoins de connaissances approfondies, ou de configurations spéciales du
serveur. Il suffit d'installer un serveur\footnote{un serveur permettant
de faire tourner une application web java comme: tomcat, jetty, ect…} et d'y
mettre l'application\footnote{sous forme de '.war'} dans le bon dossier.
De même, la base de données peu être complètement indépendante de l'application
et peu se trouver sur un serveur externe.
Le type de SGBD\footnote{Système de gestion de base de données} utilisé à peut d'importance, et il n'est pas nécessaire de créer
la base de données au préalable. L'application se charge de la créer d'elle-même
grâce à l'utilisation de l'outil Hibernate.\\
\\
%%%%% repetion, ms c'est un bou de ce que j'avais ecrit, faut voir si c utilie
La base de donnée et l'application tournent actuellement sur le même serveur, mais rien ne l'impose.
Il est donc possible de choisir de stoquer la base de donnée de l'application sur un serveur ephec, et d'avoir l'application java tourner sur un serveur "public".
Les deux peuvent également être hébergé sur des serveurs Ephec, ca nécessite l'installation de Tomcat (ou Jetty, JBoss,..), donc d'un programme ultra léger à très lourd. Multi plate-forme, et aucune configuration particulière (l'archive .war fournie avec le cd, doit juste être déposé dans le bon répertoire pour pouvoir fonctionner).
Nous préconisons cependant l'utilisation d'un Tomcat derrière un Apache, pour plus de maniabilité (par exemple pour les droits d'accès et la facilité de faire cohabiter notre application avec d'autre chose, sans risque), de perfo et de sécu. \\
\\
La base de donnée, également n'impose absolument rien. L'application communiquant à la bdd par le biés d'hibernate et de JDBC, il faut:
1. télécharger le driver jdbc par celui correspondant à la bdd, oracle en recense actuellement 221 (http://developers.sun.com/product/jdbc/drivers/)
2. configurer le fichier de config d'hibernate
3. créer une database nommé betty ainsi qu'un utilisateur ayant les droits sur cette bdd 
Hibernate se charge d'écrire toutes les tables nécessaire.  Nous avons fait plusieurs tests, avec mySql ainsi que Postgres sur des bdd complètement vierge, et aucun problèmes n'est a déclarer.

%%%%%% fin de la repetition


% peut être pas mettre ce paragraphe dans le rapport
%Nous avons choisi le langage java d'une part car nous sommes familiariser avec la programmation orienté objet.

%************************ JAVA *********************************
\section{Java}
%Cette partie est plutôt pertinente pour justifier le choix du langage Java au profit d'un autre
Le choix du langage Java c'est fait instinctivement. Celui-ci est très présent
dans le domaine du développement en entreprise. Le langage java permet de bien
structurer sont programme,il fait preuve d'une certaine rigueur, de robustesse
et offre la possibilité d'utiliser des variables typé statique. Nous avons pu
tirer avantages de ces derniers points pour établir notre application.\\
Nous avions, dans une première approche, l'intention de faire cette
application en Python. Ce Langage offre beaucoup de possibilité est aussi adapté
au type d'application que nous voulions élaborer. Il possède une structure
favorisant les bonnes pratiques, ce point étant particulièrement important pour
nous. Le langage java c'est fait plus instinctivement. Tout d'abord, les
librairies du solveur sont en java. Nous étions partis dans l'idée de faire du
binding grace à l'utiliation de SOAP entre le python et le solveur, mais
dans un souci de clarté du code, nous avons préféré rester dans le même type de
langage. L'arrivée de GWT (que nous verrons dans le point suivant) nous à aussi
conforté dans ce choix.
Il existe plusieurs types de serveur d'application java. Nous utilisons un
serveur Tomcat, écrit lui-même en java et étant multiplateforme. Ainsi
l'application peut tourner sur n'importe quel type d'architecture.


%************************ JAVA EE *********************************
% Cette partie ne me semble pas utile ici. On parle déjà du java, on dira après
% que c'est du java EE quand on parlera du serveur

%\section{Java EE}
%Pour établir notre logiciel, nous avons donc utiliser du java, sous ça forme
%entreprise édition. Ceci étant nécessaire pour l'élaboration du code partie
%serveur. l'utilisation de Google Web Toolkit nous à imposer cette utilisation,
%et nous avons du 

%************************ Google web Toolkit *********************************
\section{Google Web Toolkit (GWT)}

Nous avons choisi de travailler avec GWT pour les nombreux avantages\footnote{cfr. Chapitre GWT} que celui-ci apporte. GWT nous permet de coder la partie cliente de l'application en Java, et celui-ci génère le code javascript correspondant. Le code généré par GWT peut être adapté au différent navigateur les plus répandu à ce jour tel que Chrome, Firefox, Internet Explorer,... La partie serveur étant réaliser en java, il est plus facile pour le développeur de créer sont application dans un langage unique. GWT utilise du java EE du coté serveur, langage très puissant et ayant déjà fait ces preuves de robustesse puisque celui-ci est très répandu dans le monde professionnel. Google web toolkit propose également l'utilisation d'outils comme Gin and Juce qui feront l'objet d'un autre point. Comme sont nom l'indique, GWT est une vrai boite à outils.


%****************************** Local Storage *********************************
\section{HTML5, CSS3}
Nous avons utiliser les dernières normes de ces langages web. GWT permettant d'utiliser ceux-ci, nous avons décider d'utiliser certaine fonctionnalité intéressante qu'elle propose, comme l'utilisation d'un local storage, venu avec le HTML5, permettant de stocker les données coté client. Le CSS3 quand est à lui est utilisé pour le rendu graphique de l'application.

%******************************** Hibernate ***********************************
\section{Hibernate}
Pour la communication avec la base de données, nous utilisions l'outil Hibernate. Nous ne rentrerons pas dans les détails de cet outil, celui-ci fera l'objet d'un autre point. Nous dirons juste que Hibernate est un outil performant, permettant de représenter les tables de base de données en objet, facilitant ainsi l'utilisation des données. Il peut être utilisé avec n'importe quel type de SGBD et créé les tables automatiquement. Il possède donc de nombreux avantages et est aussi utilisé en entreprise.

%************************ Solveur *********************************
\section{Solveur}

Après analyse des différentes libraires, nous nous sommes orientez sur la
librairie xxx. Celle-ci étant beaucoup plus adapté à nos besoins, qui sont de
pouvoir gérer des contraintes sous la forme de désidérata. Elle est écrite en java et est orienter pour la problématique de la création d'horaire pour un établissement scolaire. A défaut des autres librairies étant plus orienté contraintes et non désidérata, celle-ci propose des options de configurations correspondant à nos besoins et au besoins d'un établissement tel que l'Ephec. Nous l'avons pris uniquement pour des raisons de performance et de correspondance à ce qui doit être fait, et non pas par facilité d'utilisation.

%************************** GitHub *********************************
\section{GitHub}
Puisque nous faisons ce Travail en équipe, il est nécessaire de pouvoir avoir un suivit de ce que chacun de nous fait, de pouvoir fusionner nos travaux et de garder une trace des différentes version en cas du bug éventuel. Nous avons choisi GitHub pour gérer notre projet, celui-ci étant simple d'utilisation, très performant, gratuit et permettant surtout de fusionner les différents code écrit, au sein d'une même page, de manière indépendante et intelligente.



% partie cliente de l'applicatin (gwt, gwtp, librairie,...) 
% !TeX root = these.tex

\chapter{Cadre technologique}


\section{Développement du côté client}
%*****************************Google Web Toolkit***************************
\subsection{Google Web Toolkit (GWT)}

	%************************ Présentation *********************************
GWT est un outil open source mis en ligne par Google permettant de développer des applications web avancées. En utilisant cet outil, nous pouvons développer des applications AJAX\footnote{Asynchronous JavaScript And XML} en langage Java.
Lors de l'élaboration d'une application, le cross-compiler de GWT traduit la partie cliente de l'application java en 
fichiers JS. Ces fichiers peuvent être soumis à des optimisations (obscurcir le code, séparation du JS en plusieurs fichiers, etc.). GWT n'a pas pour objectif d'être "just another library"\footnote{GWT est un ensemble d'outils qui forment une plateforme et non pas simplement une librairie facilitant le travail.} mais possède sa propre philosophie.\\
\newline
\indent
Pour programmer une application web de notre type, il faut maîtriser des outils comme l'AJAX, l'HTML ainsi que le CSS. Le problème principal de ces outils réside dans leur compatibilité entre les différents navigateurs. La façon de mettre en forme un site web n'aura pas toujours le même sur ceux-ci. Il faut prendre en compte aussi la complexité d'utilisation de ces langages de manière avancée (utilisation du DOM en HTML, etc.). Le langage JS est assez complexe d'utilisation, surtout pour l'écriture de grosses applications\footnote{Le debuggage de celles-ci étant assez fastidieux puisqu'il s'agit d'un langage interprété.}.
\newline
\indent
Pour pallier à ces problèmes, GWT a été créé. Il a été élaboré dans le but de répondre à un besoin et non pas de proposer une autre librairie standard. Il s'agit d'une vraie boite à outils qui propose des solutions de développement répondant aux besoins du programmeur.
\newline
\indent
GWT est utilisé par exemple dans des outils web de Google comme Gmail. Nous pouvons aussi citer l'entreprise Rovio à l'origine du jeu à succès Angry Bird ayant développé une version HTML5 du jeux à l'aide de GWT\footnote{cf. https://developers.google.com/web-toolkit/casestudies/}.

%************************ Principe de fonctionnement *********************************
\subsubsection{Principe de fonctionnement}

GWT possède un plugin Eclipse que nous avons utilisé dans la conception de notre solution. Il est toutefois intéressant de signaler qu'il existe des plugins GWT pour d'autres IDE\footnote{Integrated Development Environment} tels que NetBeans, JDeveloper, etc. 
Ce plugin, dans sa version Eclipse, permet d'intégrer les outils GWT à l'IDE et de faciliter son utilisation pour l'utilisateur en automatisant, par exemple, la structure des différents packages java.\\
\newline
\indent
Comme nous l'avons expliqué précédemment, le principe de fonctionnement de GWT est de pouvoir créer des applications web en java basées sur un modèle client/serveur et de sérialiser la partie cliente de l'application en JS\footnote{La partie serveur restant elle, en java}. Un autre aspect de se fonctionnement réside dans la compilation pouvant être effectuée pour un ou plusieurs navigateurs spécifiques, favorisant la compatibilité et l'homogénéité du logiciel sur chacun d'entre eux. Ces deniers chargent uniquement ce qui les concerne. En plus de ces différents aspects qu'offre GWT, ce dernier permet aussi de préciser quelles sources du code\footnote{correspondant au .CLASS généré lors de l'exécution du code Java.} prendre en compte pour un navigateur spécifique.

%************************ Mode de fonctionnement *********************************
\subsubsection{Mode de fonctionnement}
Nous distinguons deux types de fonctionnement. Le mode de développement ainsi que le mode de production.
	
\paragraph{Mode de développement}
Le mode de développement consiste à compiler les sources du projet. Celui-ci n'est pas retranscrit en JS mais est directement exécuté en byte code. Ceci afin de permettre le debuggage de l'application. GWT compile tout de même le projet en JS-HTML-CSS afin de pouvoir valider le projet.
\newline
\indent
Pour pouvoir utiliser le mode de développement, il est nécessaire d'installer au préalable le plugin de développement sur le navigateur web. Ce plugin permet de capturer les événements et actions venant du client pour ensuite les envoyer vers le serveur local lancé au moment de l'exécution du projet.

\paragraph{Mode de production}
Le mode de production correspond au code JS généré par le compilateur GWT. Le compilateur crée le JS, HTML et CSS à partir des sources du projet. Ce code correspond à l'application finale qui pourra être déployée sur notre serveur Tomcat.

%****************************Architecture de GWT******************************
\subsubsection{Architecture GWT}
L'architecture d'un projet GWT se fait sous la forme client/serveur. Nous allons analyser comment se déroule la communication entre ces deux parties. Nous distinguons deux types de communication dans l'application.
\begin{itemize}
\item Client/Client: communication entre les différentes vue de l'application
\item Client/Serveur: utilisant le protocole RPC\footnote{Remote Procedure Call} 
\end{itemize}

\paragraph{Évènement}
Afin de pouvoir gérer la communication entre les différentes vues de l'application, GWT utilise un système d'envoi d'évènements, dits \enquote{events}. Ceux-ci permettent aux vues de dialoguer entre elles.
Par exemple, une application GWT peut posséder un en-tête. Celle-ci est statique et n'est pas rechargée entre les différentes vues. Lors de l'identification, les informations de connexion peuvent être envoyées à la vue suivante pour spécifier à l'utilisateur le nom du compte sur lequel il est connecté.
Les events sont enregistrés auprès de l'eventbus. Cet eventbus peut être écouté pour récupérer les informations voulues.

\paragraph{Actions}
Les actions ressemblent aux Events, à l'exception que celles-ci sont envoyées au serveur. Elles permettent, par exemple, de faire des requêtes vers la base de données pour recueillir certaines informations. Pour rester dans le même exemple, lorsqu'un utilisateur se connecte à l'application en spécifiant sont identifiant et son mot de passe, ceux-ci doivent être vérifiés dans la base de données qui va renvoyer la liste des projets qui lui sont assignés. Cette méthode se fait à l'aide du protocole JSON\footnote{JavaScript Object Notation}/RPC.
\newline
\indent
Il est primordial de souligner que les appels se font de manière asynchrone ce qui permet de ne pas bloquer le client lors d'un appel de procédure. Cet aspect a fait l'objet d'un travail transversal tout au long du développement. 
	
%***************************** Avantage de GWT****************************
\subsubsection{Avantages de GWT}

Les avantages de GWT sont nombreux :  \\
\begin{itemize}
\item \textbf{Facilité d'utilisation}\\
Le premier avantage que nous citerons est la facilité d'installation et d'utilisation de l'outil. Ce dernier ne nécessite pas de configuration fastidieuse. En outre, cet outil offre la possibilité de coder son application à l'aide d'un langage de haut niveau. Cet aspect facilite le développement général.\\

\item \textbf{Debbugage}\\
GWT permet un debuggage rapide du code, ce dernier étant codé en java et non pas en JS qui est un langage interprété.\\

\item \textbf{Optimisation}\\
GWT permet notamment l'optimisation et l'obfuscation du code JS. De plus, afin d'améliorer la rapidité des échanges, le code JS est compressé, mis en cache et séparé en différents fichiers.\\

\item \textbf{Composants génériques}\\
GWT offre à la réutilisation différents widgets prédéfinis. Ainsi, il n'est pas nécessaire de réécrire certaines parties basiques telles qu'un bouton, une boite de dialogue, etc.  En outre, GWT offre la possibilité d'adapter ses widgets afin d'en créer de nouveaux.\\

\item \textbf{Sérialisation du code}\\
Le code Java est traduit en code JS et automatiquement adapté au navigateur de notre choix.\\

\item \textbf{JSNI}\\
GWT présente une JSNI\footnote{JavaScript Native Interface}, c'est-à-dire une interface nativement implémentée en JavaScript. Cet aspect offre la possibilité d'utiliser directement du code JS au sein même de notre application. Il est donc tout à fait possible d'utiliser des librairies externes entièrement codées en JS telle que Jquery.\\

\item \textbf{Utilisation de GinJector and Guice}\\
GinJector and Guice, n'étant pas un principe propre GWT, permet de créer une indépendance entre les différentes classes du projet. Ainsi, si nous voulons utiliser un objet ou une classe dans une autre, il n' y a pas besoin d'instancier cette nouvelle classe. Nous pouvons simplement injecter.
\end{itemize}





%**************************** Inconvénient de GWT **************************************
\subsubsection{Inconvénients de GWT}
Le principal inconvénient que nous avons pu noter suite à notre expérience personnelle est que lorsque l'application atteint un état d'avancement plus prononcé, il devient très lourd et très lent de tester son application en mode développement. La JVM\footnote{Java Virtual Machine} devant traduire le code est très lente. Il nous a fallu en moyenne 5 minutes pour charger une page, et nous avons eu fréquemment des erreurs de type \enquote{out of memory} dues à la lourdeur de la tâche effectuée par l'ordinateur en mode développement.
\newline
\indent
Pour certains types de test, nous nous sommes retrouvés dans l'obligation de déployer l'application sur notre serveur Tomcat, car la sérialisation du Java vers le JS étant très lourde, certaines parties devaient être testées directement en mode production (notamment le drag and drop, etc.). 
\newline
\indent
De même, lorsque l'application doit charger une grande quantité d'informations (grand nombre de cartons, professeurs,etc.) la page met beaucoup de temps à s'ouvrir.\\
\newline
\indent
Autre inconvénient, le temps de compilation du logiciel. Celui-ci peut être très fastidieux en fonction des paramètres demandés. 
\newline
\indent
Un autre inconvénient réside dans le nombre assez limité de widgets fournis de base dans l'application. Pour certaines fonctionnalités plus avancées nous avons dû avoir recourt à des librairies externes, bien que celles-ci ne soient pas aussi performantes (comboBox avec CheckBox, etc.).

%************************* GWT DESIGNER *******************************
\subsection{GWT-Designer}
GWT-Designer est un outil permettant de créer de manière efficace les interfaces graphiques. Ces dernières sont créées via un fichier XML\footnote{eXtensible Markup Language} et sont retranscrites en code Java. 
\newline
\indent
Ce code XML est éditable \enquote{manuellement} ou peut être créé à l'aide d'une interface de \enquote{drag and drop} ou les composants (widgets) peuvent être sélectionnés.
\newline
\indent
L'avantage d'utiliser un tel outil est double ; il permet de faire une distinction entre les différentes parties du code\footnote{Il y a plus de 13 000 lignes de code, fichiers java et xml compris.} et il permet d'utiliser le format standard XML.

%************************* GWT PLATEFORM *********************************$
\subsection{GWT - Plateform}
GWT - Plateform est un framework MVP\footnote{Model View Presenter} permettant de mieux structurer notre projet en respectant une certaine méthode. Ce framework s'incorpore directement à l'IDE et est dépendant de GWT.\\
\newline
\indent
Le MVP se base sur le MVC\footnote{Model View Controler}. Celui-ci est un design pattern fournissant une manière d'élaborer des interfaces graphiques. Il est séparé en trois parties ; le modèle de données, les différentes vues de l'application\footnote{Au sein de l'interface du client} et le présenteur\footnote{Le présenteur correspond au contrôleur du MVC}. 
\newline
\indent
La partie intéressante de ces deux modèles réside dans l'utilisation d'un contrôleur/présenteur où transitent les informations. Ainsi, l'application est soumise à leur contrôle ce qui permet d'avoir une application plus robuste. 
Dans le MVC, le contrôleur s'occupe de gérer les événements. La logique de contenu (Rendering logic) se passe directement entre le modèle et la vue. 
En MVP, cette logique est gérée par le présenteur. En définitive, rien ne transite directement entre l'interface et le modèle mais tout est soumis au contrôle du présenteur.

%*************************** LIBRAIRIE EXTERNE GWT **************************
\subsection{Libraires externes}
Pour la création d'une interface plus avancée, nous avons du nous diriger vers des librairies externes. 
\subsubsection{GWT-DND}
Nous avons utilisé GWT-DND qui comme son nom l'indique nous a permis
d'implémenter le \enquote{drag and drop} sur les cartons. Le \enquote{drag and drop} fournis par
cette librairie nous permet de capturer les Events de la souris mais aussi les Events touch permettant de garder la possibilité qu'offre GWT d'être utilisé sur des appareils mobiles. Ce qui est non négligeable à l'heure actuelle où les smartphones et tablettes ont une place prépondérante sur le marché. 
\newline
\indent
L'outil est simple d'utilisation. Pour pouvoir rendre dragable notre widget, il est nécessaire de créer un "dragControler". Les widgets s'enregistrent auprès de ce contrôleur. Une fois cette opération effectuée, il faut pouvoir définir une zone où ces widgets pourrons être dropés. Pour se faire, nous créons un nouveau "dropControler", celui-ci s'enregistre aussi auprès du dragControler pour signifier qu'il est prêt à recevoir les widgets.

\subsubsection{Smart-GWT}
Smart-GWT est un wrapper\footnote{Un adaptateur.} de la librairie JS SmartClient. Elle propose un grand nombre de widget venant s'ajouter à ceux fournis par GWT. Puisque Smart-GWT n'est qu'un wrapper de smartClient, elle ne respecte pas l'idée de base de GWT voulant que le code soit écrit totalement en Java et, ensuite, traduit en JS. Nous avons pu noter que les widgets proposés par cette librairie ne sont pas aussi réactifs que ceux de GWT.
\newline
\indent
Ces deux libraires ont été spécialement conçues pour être utilisées avec GWT. Il ne s'agit pas de librairie JS comme Jquery, mais bien de librairie\footnote{Fournies sous la forme de .JAR.} orienté GWT. Afin de pouvoir utiliser celle-ci, il est nécessaire de modifier le fichier de configuration du projet en y ajoutant les informations relatives à la librairie.


%\subsection{Serveur Linux}
%SERVEUR LINUX (Debian 64bits)
%Afin de pouvoir tester notre application, nous disposons d'un serveur tomcat tournant sur une machine %linux. L'application (sous forme de ".WAR" (en comparaison au ".jar", le "W" signifiant web)) est %l'application final possédant le code javascript (et non plus du code java comme en mode de %développement).

\subsection{Les langages web}
Le développement des navigateurs, des langages web, des possibilités que ceux-ci permettent et l'évolution de la société visant à s'orienter vers du cloud computing, nous confortent dans l'utilisation des dernières normes de ces langages. Afin de proposer une application web interactive, nous nous sommes orientés sur ces derniers langages web, même si nous n'utilisons pas toute l'étendue de ce qu'ils proposent.

\subsubsection{HTML5}
GWT nous permet l'utilisation du HTML5. L'arrivée du HTML5 permet d’accroître les performances d'une application web, en proposant des fonctionnalités plus avancées telles que l'utilisation de canevas, de balise audio ou vidéo. 
\newline
\indent
En outre, comme évoqué précédemment, une autre fonctionnalité réside dans l'utilisation d'un local storage permettant de stocker une grande quantité de données.

\subsubsection{CSS3}
Le design général de l'application a été basé sur cette dernière norme. Il est donc nécessaire d'avoir un navigateur à jour pour pouvoir profiter de l'avantage qu'offre cette norme.

\subsubsection{JS}
Même si le code JS est généré par le compilateur de GWT, elle est un élément essentiel de l'application. 
 
\subsection{Local Storage}
L'apparition du local storage (LS) avec le HTML5 offre de nouvelles possibilités aux développeurs d'application web. En effet, le local storage permet de stocker une grande quantité de données côté client. Pour bien comprendre le choix de l'utilisation de cette technologie, nous allons le comparer avec le cookie.

\subsubsection{Local Storage vs Cookies}

Tout comme les cookies, les informations contenues dans le local storage ne peuvent pas contenir d'informations sensibles\footnote{Comme, par exemple, les numéros de carte de crédit.}. Il est, cependant, nécessaire de souligner qu'aucun travail de cryptage n'a été fait sur ces données. En effet, nous supposons que les informations contenues dans notre local storage ne sont pas utiles pour les Hackers\footnote{En effet, il nous apparaît inutile de crypter le nom d'un cours.}. 
\newline
\indent
L'utilisation du local storage permet, comme les cookies, de partager les données entre les différents onglets du navigateur. Nous aurions pu nous orienter vers une solution plus simple, comme enregistrer les informations de la partie cliente dans des tableaux ou listes, mais les informations ne seraient plus partagées entre les onglets et fenêtres du navigateur. Ce choix limiterait de manière drastique l'utilisation du programme. 


\begin{table}[!h]
\begin{center}
\begin{tabular}{|l|l|l|}
  	\hline
   		& Local Storage & Cookie \\
  	\hline
  		Stockage: & 5MB & 80KB (4KB/cookie, 20 cookies/domaine) \\
	\hline  
		Bande passante : & Aucune & Envoi de données à chaque requête \\
  	\hline
  		Performance: & Data mise en cache du navigateur & Data envoyé répétitivement vers le domaine \\
  	\hline
  		Contraintes: & aucune & 300 cookies maximum \\
  	\hline
  		Rapidité: & très rapide & plus lente \\
  	\hline
\end{tabular}
\end{center}
\caption{Comparaison Local Storage et Cookie}
\label{local_storage}
\end{table}
%LOCAL STORAGE
%Capacité de stockage : 5MB
%Utilisation de la bande passante: Aucune, tout est stocker chez le client)
%Performance: Data mise en cache du navigateur
%contraintes: aucune
%rapidité: rapide (charge les données de la cache au démarrage)

%COOKIE
%Capacité de stockage: 80KB (4KB/cookie 20cookies/domaine)
%Utilisation de la BP: Envoi des données a chaque requete vers le domaine
%Performance: Data envoyé répétitivement vers le domaine
%contraintes: 300 cookies maximum
%rapidité: plus lente (serveur vérifie si le cookie existe)




Le local storage présente donc des avantages. La table \ref{local_storage} résume les différentes caractéristiques différant entre l'utilisation d'un local storage et l'utilisation de cookies.
\newline
\indent
De plus, la mise en place d'un local storage, bien qu'étant une partie fastidieuse, permet de faciliter le développement ultérieur de notre solution.

  
\subsubsection{Principe}

Le principe de ce local storage est donc de fournir un moyen de stocker et d'accéder rapidement aux données tout en limitant le trafic entre le client et le serveur. Nous ne rentrerons pas dans la structure de celui-ci, ce point faisant partie d'un autre chapitre. Les informations sont mises en cache et peuvent être supprimées à tout moment par l'utilisateur.

\subsubsection{Utilisation dans notre application}
Lors de chaque modification (comme le placement d'un carton), nous enregistrons ces changements dans le local storage ainsi que dans la base de données de manière asynchrone afin de permettre une grande rapidité de l'application. 
\newline
\indent
Dans l'hypothèse où la base de données ne serait plus accessible (ex: perte de connexion) nous avons implémenté une fonctionnalité\footnote{En version alpha au moment de l'écriture de ces lignes.} permettant de synchroniser l'état du local storage avec la base de données lors du retour de connexion.
\newline
\indent
Cependant, actuellement, une connexion internet est requise lors d'un rechargement de la page afin de se mettre à jour, ceci est indispensable pour offrir une synchronisation cohérente. Cependant une autre fonctionnalité\footnote{En version alpha au moment de l'écriture de ces lignes.} permet, si l'option est activée ou si la base de données n'est pas accessible, de fonctionner sans cette dernière, et lors de la récupération de la connexion, les données sont alors synchronisées.



\section{Développement du côté serveur}

\subsection{Java EE}

% A retravailler
La plate-forme Java Entreprise Edition est un ensemble de spécifications portées par un consortium de sociétés internationales qui offrent, ensemble, des solutions pour le développement, le déploiement et la gestion des applications d'entreprises multiniveaux, basées sur des composants objets.

\begin{figure}[!h]
    \center
   	\includegraphics[scale=0.65]{architecture_JEE.png}
   	\caption{Diagramme issu de \url{www.commentcamarche.net}}
    \label{reference1}
\end{figure}
\bigskip

%\textbf{debug : Expliquer l'image}\\
%Et dans la figure  \ref{reference1}\footnote{Diagramme issu de \url{}}
Construit sur la plateforme de Java 2 édition standard (Java SE), la plateforme Java EE ajoute certaines fonctionnalités nécessaires pour fournir une plateforme complète, stable, sécurisée et rapide de Java au niveau entreprise. 
Dans la mesure où J2EE s'appuie entièrement sur le Java, il bénéficie des avantages et inconvénients de ce langage, en particulier une bonne portabilité et une maintenabilité du code.\\
\newline
\indent
L'ensemble de l'infrastructure d'exécution Java EE est donc constitué de services (API) et spécifications tels que:
\begin{itemize}
\item HTTP et HTTPS
\item Java Transaction API (JTA)
\item Remote Method Invocation/Internet Inter-ORB Protocol (RMI/IIOP)
\item Java Interface Definition Language (Java IDL)
\item Java DataBase Connectivity (JDBC)
\item Java Message Service (JMS)
\item Java Naming and Directory Interface (JNDI)
\item API JavaMail et JAF (JavaBeans Activation Framework)
\item Java API for XML Processing (JAXP)
\item Java EE Connector Architecture
\item Gestionnaires de ressources
\item Entreprise Java Beans (EJB)
\item Java Server Pages (JSP)
\item Servlet
\item Java API for XML Web Services (JAX-WS, anciennement JAX-RPC)
\item SOAP with Attachments API for Java (SAAJ)
\item Java API for XML Registries (JAXR)\\
\end{itemize}
Dans le cadre de notre application, seule une sous-partie de ces composants a été utilisée, tels que les servelets, les JSP, JavaMail ou encore JDBC avec Hibernate.\\
\newline
\indent
L'architecture J2EE repose sur des composants distincts, interchangeables et distribués, ce qui signifie notamment :
\begin{itemize}
\item qu'il est simple d'étendre l'architecture 
\item qu'un système reposant sur J2EE peut posséder des mécanismes de haute-disponibilité afin de garantir une bonne qualité de service 
\item que la maintenabilité des applications est facilitée
\end{itemize}
\bigskip

% === avantages
Ces outils favorisent l'interaction avec, d'une part, le solveur, et d'autre part, la base de données. Cela constitue un avantage majeur de notre application. Java EE regroupe donc nativement ces libraires dans un endroit unique. Ainsi, il n'est plus utile  de devoir faire face à des implémentations complexes.
\newline
\indent
De plus, la cohérence du programme favorise la maintenance de celui-ci. Et ce choix favorise le debuggage de l'application.\\
\newline
\indent
Bien qu'ayant déjà utilisé et programmé en Java SE, nous ne nous étions jamais retrouvés confronté au Java Entreprise Edition. La différence entre ceux-ci se trouve surtout dans la complexité de leur utilisation. 
\newline
\indent
Le java SE ayant une approche plus orientée desktop. A contrario, le Java EE propose des outils avancés permettant de faire des applications internet (notamment web, mail, etc.) nécessitant un temps d'apprentissage et d'adaptation plus long. Utiliser du Java EE nous a permis d'en apprendre plus sur ce langage fort demandé en entreprise. En effet, il nous a paru important de connaître ce langage car il constitue un atout majeur pour un développeur.
\newline
\indent
Pour le log des données côté serveur, nous avons également utilisé log4j qui nous à permis de faciliter le travail de debuggage l'application.

\subsection{Hibernate}

Pour pouvoir communiquer avec notre base de données à partir de notre code java, nous avons, dans un premier temps, utilisé JDBC. Ensuite, dans un paradigme objet, nous avons  choisi d'utiliser Hibernate comme une couche supérieure à JDBC. Hibernate permettant de pouvoir transformer les tables en objets, celui-ci offre une bonne cohérence avec l'application.\\
\newline
\indent
Hibernate est un framework libre, appellé framework de  \enquote{mapping objet-relationnel} ou encore de \enquote{persistance objet des données} (voir Figure \ref{reference2}). 
\newline
\indent
Cela permet donc à la couche applicative de notre programme de traiter les données venant de la base de données comme des objets, fournissant un mapping entre le relationel et l'orienté objet.
\newline
\indent
 La base de données peut être traitée avec les avantages de l'orienté objet (comme le polymorphisme, l'héritage, etc.). Le lien entre les classes et la source physique des données  est défini au sein d'un fichier xml. Ainsi, il s'agit effectivement d'un mapping objet-relationnel.
 \newline
 \indent
Cela nous à donc permis de gérer notre base de données dans un paradigme orienté objet comme le reste de l'application.
\begin{figure}[!h]
    \center
   	\includegraphics[scale=0.65]{schema_hibernate.png}
   	\caption{Diagramme issu de \url{www.hibernate.org}}
    \label{reference2}
\end{figure}

Un autre avantage recherché par cette solution réside dans son indépendance par rapport à la base de données qui le rend entièrement portable, et cela sans aucune modification du code. Ainsi, comme évoqué précédemment, notre application peut tourner sur 221 base de données différentes.  En effet, notre application, après avoir configuré ses crédentials dans le fichier de configuration de Hibernate, se chargera de créer l’entièreté des tables et la structure de la base de données.\\
 \newline
 \indent
Théoriquement il aurait donc été possible d'envoyer directement un objet "hibernate" (représentant par exemple une table de notre base de données) à GWT donc à l'utilisateur côté client. Cependant, cet aspect reste théorique comme stipulé dans la documentation de GWT\footnote{\url{http://developers.google.com/web-toolkit/articles/using\_gwt\_with\_hibernate}}.
\newline
\indent
En effet, une SerializationException est levée à chaque fois qu'un type transféré via RPC n'est pas sérialisable. La définition de la sérialisibilité signifie ici que le mécanisme RPC - GWT sait comment sérialiser et désérialiser le type de byte code au format JSON et vice-versa.
\newline
\indent
Le problème vient du fait que Hibernate modifie les objets afin de les rendre persistants\footnote{Pour être exact, c'est la librairie Javassist qui se charge de réécrire le byte code de ces objets pour les rendre persistants.}. Au moment du transfert de l'objet, une serialisation est tentée, mais l'objet n'étant pas le même (car modifié par javassist), il ne peut être sérialisé par RPC-GWTP.
\newline
\indent
Pour répondre à cette difficulté majeure, il a été nécessaire d'ajouter des classes de type DTO\footnote{Data Transfer Object}, qui, étant sérialisables, ont pu servir de communication entre la partie cliente et serveur.\\
\newline
\indent
À noter que Hibernate propose son propre langage de requête HQL\footnote{Hibernate Query Language}, syntaxiquement proche du SQL, qui lie vision objet et vision relationnelle\footnote{\url{https://docs.jboss.org/hibernate/orm/3.3/reference/en/html/queryhql.html}}. 

%Hibernate possédant sont propre langage de requête HQL \footnote{Hibernate Query Language}, d'apparence similaire au SQL, mais pleinement orienté Object et comprenant des notions comme l'inhéritance ou de polymorphisme\footnote{\url{https://docs.jboss.org/hibernate/orm/3.3/reference/en/html/queryhql.html}}. Nous n'avons pas pu tirer pleinement avantages et n'étant pas habituer à ce genre de pratique, il a été parfois déroutant d'utiliser un langage mélangeant le SQL et les notions d'orienté objet.



% Comment nous avons travailler (gitHub, skype, egit,...)
% !TeX root = these.tex
\chapter{Méthodologie de conception}
Un programme informatique, comme n'importe quel projet, nécessite une bonne gestion pour pouvoir assurer son bon développement. 
D'une part, il faut faire face aux difficultés liées au travail d'équipe, d'autre part, il faut gérer le projet en tant que tel.

  \section{Git pour le développement collaboratif}
  La création d'un logiciel en équipe implique la gestion de cette équipe ainsi que le partage et l'échange des
  informations entre chaque membre de celle-ci.
  
  L'équipe doit travailler de façon coordonnée. En effet, elle doit s'échanger l'état
  d'avancement du travail ainsi que les résultats obtenus. Dans cette visée, nous nous sommes tournés vers le programme de gestion de versions Git développé par Torvald et, notamment, utilisé dans le cadre du développement du noyau Linux.\\
\newline
\indent
  Git est un logiciel de gestion de version décentralisé. Il permet de
  travailler à plusieurs sur un même projet. Il gère lui-même l'évolution du
  contenu en fusionnant les changements sans perte d'information. En outre, il garde
en mémoire toutes les versions du code. De plus, il est libre et
  adapté aux grands projets.\\
  \newline
  \indent
  Afin de permettre un stockage de notre code, nous avons de nous tourner vers un dépôt GitHub\footnote{GitHub est présenté précédemment.}.
  
  

  \section{Logiciel de suivi de problème}
  
  En plus de reposer sur Git, GitHub offre un logiciel de suivi de problème, aussi appelé  bugtracker. Un logiciel de suivi de problème est en quelque sorte un journal des problèmes classés par type.  C'est un outil particulièrement utile pour le développement en équipe. Il donne un aperçu clair des bugs et de leurs états de résolution.

  \section{Méthode de travail}
Pour ce travail, il a été important d'avoir une méthode rigoureuse. Nous avons donc scindé le travail en plusieurs parties, sous forme de blocs à développer.\\
\newline
\indent
Pour chacun de ces blocs, nous avons noté les problèmes de conceptions et, ensuite, nous avons analysé comment les résoudre. Par exemple,  la difficulté d'implémentation du système d'échanges asynchrones ou, encore, l'absence de support, par GWT, du filtre de cartons ont dû être analysées et ont été l'objet de solution respective. \\
\newline
\indent
Nous avons suivi la méthode RUP\footnote{Rational Unified Process} basée sur UP, définissant un moyen de travailler sur des applications de type orienté objet. Cette dernière fournit de nombreux avantages comme le rôle de chaque acteur du projet, le cycle de vie de l'application, etc. Nous nous sommes donc basés sur cette méthode pour élaborer notre travail dans de bonnes conditions.
\newline
\indent
Nos besoins étant de trouver une méthode de travail en équipe et d'avoir une bonne gestion de projet, nous ne rentrerons pas dans les détails de cette méthode. Toutefois, nous pouvons dire qu'elle nous à été bénéfique pour l'élaboration de notre application. Certains aspects de cette méthode comme la gestion de projets ont pu être utilisés via GitHub et son bugtracker. \\
\newline
\indent
Nous avons conceptualisé notre application dans une optique d'extension. Dans cet objectif, rien n'a été fait statiquement afin de garantir l'évolution future de ce projet. C'est pour ces raisons que certains choix on été effectués, nous rendant la charge de travail plus lourde mais possédant des avantages non négligeables pour la bonne évolution de l'application. L'étendue et le potentiel d'un tel outil-logiciel sont grands, surtout pour un établissement scolaire. Par conséquent, nous nous sommes donc confortés dans cette optique.

\section{Communication}

Une bonne communication interne a dû être assurée afin de pouvoir prendre des décisions sur certaines parties du projet. Bien qu'ayant défini la part de travail de chacun, il est important de bien comprendre ce qui a été fait par l'autre. Il faut aussi pouvoir tirer avantage d'un travail à deux, surtout dans le cadre de réflexion concernant la méthode à utiliser. Par exemple, les choix devant être établis en priorité pour élaborer certaines parties de l'application ou pour régler les bugs, etc. \\
\newline
\indent
Nous avons utilisé des logiciels de VOIP et nous  nous sommes réunis régulièrement afin de discuter des différents points. Ainsi, il était plus aisé de parler des problèmes rencontrés et de trouver, ensemble, une solution à ces problèmes. Certaines tâches ont donc été effectuées séparément et d'autres conjointement.

\section{Apports périphériques}
Nous avons, dans le cadre de nos cours à l'EPHEC, eu l'occasion d'utiliser certaines méthodes permettant de bien clarifier un programme telle que la méthode QQOQCP\footnote{Quoi? Qui? Où? Quand? Comment? Pourquoi?}. Nous avons donc tiré parti de cet apprentissage pour introduire la problématique de notre travail. 
\newline
\indent
Ainsi, en reposant sur les différentes méthodologies existantes, nous avons beaucoup appris sur les problèmes pouvant intervenir dans l'élaboration d'une application (que ce soit web ou desktop). Il est bien entendu impératif de se rendre compte de ces différents points pour pouvoir avoir une cohérence au sein d'une équipe. Il est fréquent pour un développeur de devoir travailler en équipe dans une entreprise. Ainsi, cette approche nous donne l'opportunité d'être mieux préparés au monde professionnel.



    
%        * mettre ici la façon dont vous avez \enquote{fait} le pgm* \\
%    
%    cad comment vous avez fait en pratique : on a ciblé nos besoins (citer les besoins) et nous nous sommes
%    dirigé vers le type de conception/méthodologie MACHIN(*mettre une référence*) qui consiste à faire de l'essai %erreur et être à la bourre 
%sans savoir où on va ni n'ou on vient. MAIS, comme on est pas des poulets, on a fait un planning. Le bugtracker nous a %également été très utile.
%
%Ce paragraphe n'est pas nécessaire, mais dans l'optique où votre promoteur aimera savoir ce que vous avez glandé %pendant l'année, c'est ici qu'il 
%faut lui prouver que 1> vous aviez une bonne méthodologie de travail (important) 2> vous avez pas glandé = pensé à tel %ou tel problème futur, telle ou telle option future, tel truc plus compliqué à faire mais qui permet ceci ou cela.
%Attention, QUE du blabla, rien de technique (vu qu'on l'explique que après)


% organisation du code (avec des pseudo diagramme uml, explication local storage, rpc , % guice,...)
% !TeX root = these.tex

\chapter{Structure du code}

% schéma de la base de données (bdd de stockage)
\section{La base de données}

% le solveur qu'on a choisi, comment ça marche
% !TeX root = these.tex
\chapter{Solveur}
% le but ici n'est pas d'énumérer toutes les étapes par lesquelles nous sommes passé ainsi que toutes les essais infructué effectué, car ce serait long et inninteressant. Nous allons nous concentrer sur la méthode actuelle utilisée en tantant de justifié son choix. Pour pouvoir expliquer son choix et sont utilisation, nous allons devoir revenir sur les explications plus générale de la programmation par contraites et d'autre thechniques OR (operation research).  Pour plus d'information sur le sujet, nous invitons le lecteur, à XXX, ainsi qu'à la thèse de M qui fut de loins l'ouvrage qui nous a le plus éclairé quand la à la problematique de xx

%On explique d'ou il vient, ce que c'est, par qui ca a été créé

%Après, on explique l'implémentation dans notre logiciel. On en est ou? Les dificulté rencontré? Ce qu'il manque pour que ça soit super cool

%Ensuite, on explique un peu comment ça marche 

%On peut terminer en concluant sur le temps d'apprentissage, les problèmes relatif a la communcation entre client-serveur... et que, dans une prochaine mise à jour, il serait possible d'avoir un truc vachement plus balaise

Pour pouvoir apporter un réel avantage à notre application, comparé à la méthode actuelle de création d'horaire (i.e. manuel), il était nécessaire de lui rajouter une certaine "intelligence". 
Une partie de cette "intelligence" est prise en charge du coté client, qui comme décrit dans une précédente section, se charge d'assister l'utilisateur lors de la création de son horaire.
\newline
\indent
Cependant, ce support est minimaliste et une recherche plus poussée est effectué du coté serveur à l'aide d'outil de programmation par contraintes et de "recherche opérationnelle".  Ce solveur sera ensuite amené à nourrir le solveur minimaliste du coté client et proposer un certain pilotage de ce solveur au travers de l'application cliente. 
\newline
\indent
Pour arriver à ces résultats, nous nous sommes tournés vers des outils de programmation par contraintes et de recherche opérationnelle. 
Pour comprendre notre choix de librairie ainsi que son utilisation, il est important de revenir sur les principe fondamentaux de la programmation par contraintes.


\section{Rappels théorique}

La programmation par contrainte un paradigme de programmation \footnote{paradigme} ayant pour but la résolution de problèmes combinatoire en
descrivant plutot le but recherché que la méthode utilisé, c'est pourquoi la programmation par contrainte es une forme de programmation déclarative.
« Constraint Programming represents one of the closest approaches computer science has yet made to the Holy Grail of programming: the user states the problem, the computer solves it. »
— E. Freuder--citation--

dans le cas qui nous concerne, la problème de création d'horraire, est un cas typique de CSP (Constraint Satisfaction Problem), et comme tout csp peut se définir comme suit:
*** definition formelle ***
Dans notre cas, les variables sont, se qu'on va appeller des "activity", et correspondent à un cours ou à tous type d'activitées pouvant avoir lieu à l'Ephec (il s'agit, dans la méthode traditionnel des cartons physique qui seront manipulé lors de la création d'un horraire.)

Le domaine de valeurs que peuvent prendre ses variables, ses activity, sont un local et une periode de la semaine.  Ca fait un domaine à deux dimentions (Lieu X Periode). Dans l'idéal, il faudrait prendre le choix des professeurs comme troisième dimension, mais ce n'est pas le cas dans notre impémentation actuelle. 

Les contraintes deviennent alors les relation mathématiques reliant les activity.
On a tenté de faire des listes exaustives de ses contraintes, mais seule une sous partie est actuelement pris en compte:



\subsection{Choix de librairies}
Le solveur (celui du coté serveur), est indépendant de tout, absolument tout. Il est écrit en java, et repose sur une librérire de csp. Il réside donc du coté serveur, dans le servelet de l'application, c'est donc le meme pour tout utilisateurs se connectant au  site. Il ressevra l'information et l'enregistrera xxx


Apprès bpc de recherches, cette librairie est de loins la plus adapté et à jours (la dernière version datant du 20 juin, est d'ailleur celle utilisé par le programme).  Ca permetra un dévellopenment du solveur, independament de la bdd ou de l'interface.
Les possibilités de cette librairie combiné à nos choix d'infrastrcuture sont énorme, et à notre grande tristèsse, ne sont pas utilisé au maximum de sa puissance..  seule la surface à pu être implémenté, faute de temps.

\begin{center}
\fbox{\begin{minipage}{\linewidth}
\textbf{Definition 1} $(CSP).$ A constraint satisfaction problem $(CSP)$ is a triple $\Theta$ = (V,D,C), where
\begin{itemize}
\item V = \{$v_1, v_2, ... , v_n$\} is a finite set of variables,
\item D = \{$Dv_1, Dv_2, ... ,Dv_n$ \} is a set of domaines (i.e., $Dv_1$ is a set of possibles ...
\item C = \{$c_1, c_2, ..., c_m$\} is a finite set of constraints restricting the values that the variables can simultaneously take.\\

\textbf{Definition 2} $(assignment)$. Let $\Theta$ be a CSP, an assignment of the variables\\ from V is $\eta$ $	\subseteq$ \{$v$/$a$|$v$ $\in$ V \large \&\normalsize \mbox{} a $\in$ Dv\} where $\forall$ $v$/$a$, $w$/$b$ $\in \eta$ $v$ = w $\Rightarrow$ $a$ = $b$.\\
An element $v$/$a$ of $\in$ means that variable $v$ has assigned value $a$.\\
An assignement is complete if |$\eta$| = |V| ( i.e., all variables are assigned).\\

\textbf{Definition 3} $(solution$ $to$ $CSP).$ A solution to the constraint satisfaction problem $\Theta$ is a complete assignement $\sigma$ of the variables from V that satisfies all the constraints.
\end{itemize}
\end{minipage}}
\end{center}


% se trouve pour le moment, les differents chapitre de relexion
% concernant par ex la secu, la perf, etc
% !TeX root = these.tex
\chapter{Discussions}
\section{Choix concernant la performance}
La question de performance pour un logiciel est un point très important à prendre en compte. En effet, il faut que le programme puisse fournir une bonne fluidité et que celui-ci soit agréable à utiliser. Pour ce faire, nous avons opté pour une solution permettant d'offrir ces performances. L'architecture de l'application, les fonctionnalités utilisées ainsi que 

\section{Choix concernant la sécurité}

Du fait de nos choix de conception, tel que GWT pour générer le JavaScript ou
Hibernate pour abstraire la base de donnée ou le choix des RPC pour la
communication client-serveur,etc.  Nous pensons avoir créé une application \enquote{de
base} très sécurisée  et qu'il serait tout à fait envisageable de faire tourner notre
application sur un serveur externe.

(xxx) xss, csrf, sha256, check (basique) de la complexité du mot de passe, ainsi
qu'un catcha très basique aussi pour eviter des bots.

La secu du coté serveur à également été pris en compte, mais, logiquement, en
moins poussé (quoi que).  A savoir: l'application tourne sur une debian *stable*
(à savoir Squeeze), n'ayant que peu de services installé.
Tomcat, notre conteneur d'applet, est lui aussi à jours, et en version stable,
il ne bénificie d'aucun droit root, ainsi donc, (contrairement à une autre
application tournant sur Windows par exemple), il n'a pas le droit d'emmetre sur
le port 80. Pour garder l'application safe et peformante, une redirection de
son port d'origine est effectué par netfiller sur le port 80.

Nous avons pas poussé plus loins la sécu du coté serveur, étant donnée qu'elle
est destiné à fonctionner sur d'autre machine, mais en situation réel, il serait
jusdicieu de mettre en place un liens https.  En situation réel, nous
préconisons xxx (parfeu, strapping, vm,..)

N'étant pas infaillble, la mise à jour (très aisée pour notre type
d'application), est nous semple -t-il, une part importante dans la sécurité du
système. (cela probablement accentué par la mise de notre code source sous une
license ouverte (gpl3) )



% Amélioration envisagable pour le logiciel
% !TeX root = these.tex
\chapter{Perspectives}

\section{Possibilités d'amélioration du programme}

\subsection{Mode comparaison}
Cette partie étant un de nos objectifs principaux. Il est question de pouvoir créer à la
\enquote{volée} les colonnes représentant plusieurs professeurs, locaux ou classes, tout
 en gardant, pour les lignes, les périodes. Malheureusement nous n'avons pas eu le temps de l'implémenter. Un autre problème fondamental ayant retardé son élaboration est qu'il n'est  possible de rendre une région ne faisant pas partie du DOM (typiquement rajouté 
en JS, donc d'une manière non-statique) comme étant une région \enquote{dropable}.
Cela n'étant pas possible, la solution idéale nous est encore inconnue. Il faudrait probablement rendre l'entièreté du tableau comme \enquote{dropable} car actuellement uniquement les cases du tableau sont dropables.

\subsection{Solveur}
Ce fut également un point assez frustrant, étant donné nos recheches 
sur le sujet, et surtout la trouvaille de cette librairie ``miraculeuse'' écrite 
par Muller, et qui, en plus d'être en gpl et fortement maintenue (dernière 
version date d'il y a 2 semaines), elle nous paraît très performante et d'une 
rare adéquation.  C'est donc avec beaucoup de tristesse dans l’âme (:p), qu'on n'
utilise qu'une fraction de ses possibilités.  C'était tout l'avantage de notre 
application (et de sont architectur client-server) et ce n'est pas utilisé a son
full potentiel.  C'est donc la première amérioration que devra subir notre
application, car tout est en place pour pouvoir en tirer un avantage certain.

\subsection{Droits}
Un des buts de cette architecture était de pouvoir avoir plusieurs comptes liés à
un projet, avec des droits différents. Imaginons par exemple, des professeurs qui pourraient
 éditer leurs désidérata, ou la possibilité pour un utilisateur de pouvoir suivre l'évolution d'un horaire, sans pour autant pouvoir modifier celui-ci.

\subsection{Mode hors ligne}
Nous nous basons entièrement sur le local storage, les requêtes vers le serveur ne 
servent qu'a se connecter, charger le projet et enregistrer des modifications.
Actuellement, il est possible de continuer la création d'un horaire chargé en 
mémoire sans la nécessité d'une connexion, cependant, pour pouvoir pleinement 
utiliser cette fonctionnalité, il est nécessaire de rajouter certaine fonctionnalité, telle 
qu'une resynchronisation des modifications faites en local avec la base de données lorsque la 
connexion est retrouvée. La possibilité, pour l'utilisateur, de reprendre un projet toujours conservé dans la mémoire cache.

\subsection{Upload à partir d'une base de données}
le fichier actuellement utilisé pour l'upload des attributions est le résultat d'une requête SQL. Dans un souci de clarté et de facilité pour l'utilisateur, il est plus évident de devoir rentrer un fichier manuellement étant donné qu'un autre fichier des locaux doit être également fourni. Pour qu'un upload puisse se faire de manière complète, il faudrait que la liste des locaux avec ce que ceci comporte soit également enregistrée dans une base de données afin de ne pas perdre l'utilisateur sur deux méthodes différentes d'envoi et de pouvoir ainsi garder une certaine cohérence du logiciel.

\subsection{Modifications de données}
actuellement il n'est pas possible de modifier toutes les données intervenant 
dans l'application (notamment la création d'un prof, local, etc.) Il faudrait donc avoir la possibilité d'ajouter certaines informations directement via l'interface du logiciel, afin de prévenir de tout oubli, ou rajout de dernière minute sur le nombre de classes, etc.

\subsection{Upload de contraintes}
L'idéal serait de permettre à l'utilisateur d'uploader son fichier de contraintes (sous forme de fichier XML par exemple), ou de pouvoir les entrer manuellement via l'interface. N'étant pas l'objectif de base de notre travail de fin d'étude, celui-ci étant plus orienté sur la programmation par contraintes à proprement parlé et l'intégration de cette programmation dans un outil d'aide à la création d'horaire, et donc l'élaboration de ce logiciel, ceci limite l'utilisateur de l'application aux uniques configurations que nous aurions choisesi au préalable. Il faudrait donc dans l'idéal permettre à l'utilisateur de pouvoir manipuler lui même les contraintes.

% nos recommandation sur l'utilisation, le fichier xls,...
%% !TeX root = these.tex
\chapter{Recommandation}

% !TeX root = these.tex
\chapter*{Conclusion}

Dans le présent rapport, nous avons vu comment répondre au problème réel et récurrent de la création d'horaires dans le contexte académique. Pour répondre à cette problématique, nous avons dû effectuer un large travail sur trois points en particuliers ; la programmation par contraintes, une solution client/serveur et, enfin, un travail collaboratif en équipe.
\newline
\indent
Pour la programmation par contraintes, nous 

-- on conclut dans le foin :) --\\

->  point cool perso: utilisre des nouvelles technologies \\

->  ce projet nous tien à coeur \\
non, il n'est pas encore completement utilisable dans le sans que nous voudrions.
Il est possible de créer horraire, mais les avantage qu'offrent l'outils informatique ne sont eux pas fonctionnel. Notre solution, actuelement offre donc moins d'avantage que le faire en version papier.  Nous croyons cependant bcp dans notre programme et pensons sincèrement, que mme si c'était pas les les plus simple, les décisions que nous avons prise peuvent permettre énormément de choses.  \\

-> nous avons été supris par la dificulté (dans le sans qu'un petit scout est
surpris par la nuit :p ), et il à été très frustrant de devoir faire face à
d'autre problèmes qu'on avait pas imaginé et qui nous ont dévié de nos objectif
premier.  Heureusement ce fut des problème interessant qui nous ont bcp appris,
que ca soit sur la gestion du code, gestion de notre ``équipe'' (fair ce tfe à
deux était une exelente expérience. ca nous à permit de fournir du code plus
propre, car relu, des idées plus aboutie, \ldots Notre gestion, et la grosseur
de la tâche ainsi que sa ``segmentation'' nous aurais permi d'éfectuer ce tfe
avec 2fois plus de participant, sans qu'ont ne se ``marche sur les pieds'',..)


\backmatter

\newpage
\bibliography{official}
\bibliographystyle{apalike-fr}

%\newpage
%\listoffigures
%
%\newpage
%\listoftables

\renewcommand{\thesection}{\Alph{section}}
\setcounter{section}{0} 

%% !TeX root = these.tex
\newpage
\chapter*{Annexes}

\setcounter{page}{1}
\pagenumbering{roman}
%\pagenumbering{Roman}
%\pagenumbering{Arabic}




\section{Page de connexion et d'inscription}
\label{annexe/espace_nom}

\begin{figure}[!h]
	\begin{center}
		\includegraphics[width=12cm,height=6cm]{login.png}	
		\caption{page de connexion}
	\end{center}
\end{figure}

\begin{figure}[!h]
	\begin{center}
		\includegraphics[width=12cm,height=6cm]{subscribe.png}
		\caption{page d'inscription}
	\end{center}
\end{figure}

\newpage

\section{Page des projets}
\label{annexe/espace_nom}

\begin{figure}[!h]
	\begin{center}
		\includegraphics[width=19cm,height=12cm,angle=90]{ProjectPage.png}
		\caption{page des projets}
	\end{center}
\end{figure}\newpage

\newpage

\section{Page principale I}
\label{annexe/espace_nom}

\begin{figure}[!h]
	\begin{center}
		\includegraphics[width=19cm,height=12cm,angle=90]{MainPageClean.png}
		\caption{page principale vide}
	\end{center}
\end{figure}

\newpage

\section{Page principale II}
\label{annexe/espace_nom}

\begin{figure}[!h]
	\begin{center}
		\includegraphics[width=19cm,height=12cm,angle=90]{MainPagePlacedCard.png}
		\caption{page principale carton placé}
	\end{center}
\end{figure}

\newpage

\section{Page principale III}
\label{annexe/espace_nom}

\begin{figure}[!h]
	\begin{center}
		\includegraphics[width=19cm,height=12cm,angle=90]{MainPageColorTab.png}
		\caption{page principale solveur client}
	\end{center}
\end{figure}

\newpage

\section{Card filter}
\label{annexe/espace_nom}

\begin{figure}[!h]
	\begin{center}
		\includegraphics[width=6cm,height=12cm]{CardFilterAllCard.png}
		\includegraphics[width=6cm,height=10cm]{CardFilterBouterfa.png}
		\caption{présentation des cartons (non) filtrés}		
		\includegraphics[width=6cm,height=8cm]{CardFIlterChoice.png}
		\caption{filtre avec multi-sélection}
	\end{center}
\end{figure}

\newpage

\section{Schéma entité association}
\label{annexe/espace_nom}

\begin{figure}[!h]
	\begin{center}
		\includegraphics[width=15cm,height=19cm]{EA.png}
		\caption{concept de schéma de la base de données}
	\end{center}
\end{figure}
Celui-ci étant un concept, car nous avons travaillé avec des classes via le framework Hibernate.




%\makeback
\end{document}
