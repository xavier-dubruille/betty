% !TeX root = these.tex
\chapter{La base de données}
la base de données se veut, délibérément, extrêmement simple et minimale. Le but n'étant pas de gérer l'information mais plutôt de la stocker.

Par exemple, nous ne nous soucions pas de données propre à l'année, la section, l'implantation, etc.,  Nous nous focalisons uniquement sur les attributions. Les informations qui les composes se doivent d'être séparées en différentes parties (par exemple, les professeurs, les différents groupes, etc.). Les données sont statiques et ne seront pas vouées à être modifiées.

L'objectif n'étant pas de faire une base de données relationnelle dans les règles de l'art, mais d'avoir les données rassemblées en un point fixe (comme les classes (Année, Section, groupe)) et de pouvoir accédé aux données de la manière la plus rapide qu'il soit. Nous avons donc opté pour ce système de stockage, évitant les cout imputer au passage entre les différentes tables pour récupérer l'ensemble des informations dont nous avons besoins, et permettant de garder une logique sur ce que doit représenter une attribution.

%Xavier, peut tu retravailler cette partie ;-)
Nous pouvons noter une table plus particulière de la base de données, la table activityState. Celle-ci correspond à l'état d'un carton (idéalement à un instant donné), c'est à dire, si ce carton est placé, à quel endroit et à quel moment il a été placé.  Le but est de pouvoir offrir des retours en arrière ou encore, une analyse des mouvements effectués, comme par exemple en sélectionnant un carton et voir les états précédent de se dit cartons (mais cela n'est pas fait dans la version actuel du programme) ainsi que de priviligier l'ajout à lecture (car il faudrait faire une recherche et un update à chaque pose de cartons), et également permetre de régler des litiges pouvant survenir (utilisation synchrone du programme par exemple).  Il rend également la gestion des différentes "Instances" bcp plus simples.  L'inconvénient, c'est qu'il faut faire une recherche un peu plus longue lors du chargement du projet car il faut selectionner le nernier état du carton. Egalement, il est possible que cette table pourrait atteindre de grosse tailles, et il est necessaire, une fois le projet cloturé, de supprimer les données inutiles. 

Les "instances" sont un set d'états des différents carton au sein d'un même projet. Les instances étant des sous partie de ce projet, chaque carton placé ne sera pas le même suivant les différentes instances.

pour la gestions des contraintes, il est également necessaire de les sauvegarder, et seulement un set non exostif de contraintes est possibles d'etre enregistré (et traité dans notre programme). Néanmoins, ce set est relativement large et permet de "nommer" la plus part des contraintes.  Il s'agit principalement de "préférences".  Préférences qui pourront être "hard" ou "soft" (un indiquateur est possible pour y mettre plus de grénulosité, mais il n'est actuelement pas pris en compte par notre solveur).  préférence applicable sur un Moment (jour/period) ou sur un(des) local(aux).  préférences dont la sources peuvent être prof, cours, groupe, local.

Avec ca, il est possible de 


