\section{Méthodologie de l'écriture d'un programme à plusieurs}

La création d'un logiciel en équipe implique la gestion de cette équipe ainsi qu'une bonne méthodologie de conception.
L'équipe doit travailler de façon coordonnée ; elle doit s'échanger l'état
d'avancement du travail ainsi que les résultats obtenus. Pour ce faire, il faut
de bons moyens de communication.

La création d'un logiciel nécessite également une méthode de programmation et une architecture sous-jacente. 
Un grand nombre d'architectures et de méthodes existent déjà. Nous 


\subsection{nom de votre méthodologie générale}

\subsection{GIT et partage des données}

Git est un logiciel de gestion de version décentralisé. Il permet de
travailler à plusieurs sur un même projet. Il gère lui-même l'évolution du
contenu en fusionnant les changements sans perte d'information. Il garde
également en mémoire toutes les versions du code. Il est libre, gratuit et
particulièrement facile à utiliser.
Git nous a permis d'écrire à deux sur un même code. Il gère lui-même
l'assemblage des modifcations et propose des outils pour résoudre les conits.
Comme il garde toute trace de changement effectué dans le code, il permet
d'en récupérer n'importe quelle version.
Nous avons mis le programme sur un dépôt Git sur internet. Ainsi, dans l'avenir,
les personnes qui désireraient reprendre le programme pourront le faire di-
rectement.


\subsection{Bug trackeur}

